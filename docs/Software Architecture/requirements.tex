\documentclass[a4paper,10pt]{article}

\usepackage[margin=2cm]{geometry}
\usepackage{graphicx}
\usepackage{hyperref}
\usepackage[all]{hypcap}
\usepackage{tabu}
\usepackage[title,titletoc,toc]{appendix}
\usepackage[english]{babel}
\usepackage{fontspec}
\usepackage[style=authoryear,backend=biber,sorting=nyt,dashed=false,urldate=long,abbreviate=false]{biblatex}
\usepackage{float}
\usepackage{fancyhdr}
\usepackage{microtype}

\addbibresource{references.bib}

\setlength{\headheight}{15.2pt}
\pagestyle{fancy}
\lhead{}
\chead{}
\rhead{\bfseries Software Architecture Specification}
\lfoot{Team Echo}
\cfoot{COS301 Software Engineering}
\rfoot{Page \thepage}
\renewcommand{\headrulewidth}{0.4pt}
\renewcommand{\footrulewidth}{0.4pt}

\setlength{\parindent}{0pt}
\setlength{\parskip}{1ex plus 0.5ex minus 0.2ex}

\frenchspacing

\title{\includegraphics[width=12cm]{Eeufeeslogo.jpg} \\
       Software Architecture \\
       Research Paper Management System \\
       \vspace{0.5cm}
       University of Pretoria \\
       \vspace{1.0cm}
       }

\date{} 
\author{Team Echo\\
	\vspace{0.5cm} \\
	\begin{tabu} to \textwidth { X[l] X[l]}
		\hline
		\textbf{Surname, First Name (Initial)}	& \textbf{Student Number}	\\ \hline \hline
		Bode, Elizabeth (EF)			& 14310156		\\ \hline
		Bondjobo, Jocelyn (JM)		& 13232852		\\ \hline
		Broekman, Andrew (A)		& 11089777		\\ \hline
		Loreggian, Fabio (FR)			& 14040426		\\ \hline
		Schutte, Gerome (GC)		& 12031519		\\ \hline
		Sefako, Motsitsiripe (MG)		& 12231097		\\ \hline
		Singh, Emilio (E)			& 14006512		\\ \hline
		\hline
	\end{tabu}}

\DefineBibliographyStrings{english}{%
urlseen = {\mbox{Accessed} on},
urlfrom = {[Online]},
in = {In},
}
\DeclareFieldFormat{url}{\bibstring{urlfrom}\space\url{#1}}
\DeclareFieldFormat{urldate}{\addcomma\space[\bibstring{urlseen}\space#1]}
\addto\captionsenglish{
  \renewcommand{\contentsname}
    {Table of Contents}
}

\begin{document}
\maketitle
\thispagestyle{empty}
\clearpage

\newpage
\pagenumbering{roman}
\thispagestyle{empty}
\tableofcontents
\clearpage

\newpage
\pagenumbering{arabic}

\section{Architecture requirements}
\subsection{Architectural scope}
\subsection{Quality requirements}
\subsection{Integration and access channel requirements}
\subsection{Architectural constraints}

\section{Architectural patterns or styles}

\section{Architectural tactics or strategies}
 
\section{Reference architectures and frameworks}
 
\section{Access and integration channels}

\section{Technologies}
\subsection{Backend System}
\subsubsection{Programming Languages}
\subsubsection{Frameworks}
\subsubsection{Libraries}
\subsubsection{Database System}
\subsubsection{Operating System}
\subsubsection{Dependency Management and Build Tools}

\subsection{Web Interface}
\subsubsection{Programming Languages}
\subsubsection{Frameworks}
	\begin{itemize}
	\item AngularJS\\
		AngularJS is a JavaScript web application framework maintained by Google and volunteers, provding developers with a Model-View-Controller(MVC) and Model-View-ViewModel (MVVM) architectures to developer Single Page Applications (SPA). As the community has grown around AngularJS various plugins and components have been provided which eases the developement of rich internet applications, espesically with excellant support from AngularJS itself to support RESTful applications.

		The reason that AngularJS was chosen, is that it is a mature code base, with a large support community and which modern software engineering tools support to ensure a quality application is delivered to the end client. This allows one to do unit testing with ease, allow for the integration of a frontend end dependency management system such as bower and using modern build tools such as Grunt, in the end deliver a fast, stable, secure and efficient modern single page application.
	\end{itemize}

\subsubsection{Libraries}
\subsubsection{Database System}
\subsubsection{Operating System}
\subsubsection{Dependency Management and Build Tools}

\subsection{Android Client}
\subsubsection{Programming Languages}
	\begin{itemize}
		\item Java\\
			We will be writing a native Android application and hence we are required by the constraints of the Android system to utilize the Java programming language. In terms of Android using Java, Android relies heavy on the following Java concepts:
			\begin{itemize}
				\item Platform Independence\\
					After the developers have developed there applications in a selected programming language, the programmer often needs to compile there source code. The process of compiling source code is the conversion of the code written by the developer into a language that a certain device can understand. 
					
					It is important to note that different devices, understand different machine level langauges, and as such the developer is often forced to recompile the application for different devices understanding different machine languages. Using Java, Java source code is converted into a language called "bytecode", which is then interpreted and exectued by the Java Virtual Machine (JVM), which operates much like a physical CPU does.
					
					Using this ability of Java, it allows developers to compile there application once, and have it available on multiple different devices, by not needing to worry what machine language powers the given device. This gives developers the ability for the application to run on Android phones, tablets to even the latest toaster, car or water-bottle powered by Android.
					
				\item Secure Virtual Machines\\
					Since Java execute in a virtual machine, Android has taken advantage of this ability, and runs each application in its own virtual machine as seperate Linux users. In this way Android abstracts the execution of machine language away from the physical hardware, into a virtual machine which encapsulates and contains the bytecode and which can be more easily managed should applications not fulfill certain constraints, such as consuming too much power.
				\end{itemize}
		\item eXtensible Markup Language (XML)\\
			Android makes extensive use of XML to assist it in creating user screens in a process called inflation, storing string literals which allows for internationalization based on user preferences, and the selection of media such as image, video and audio based on the devices capablities, such as screen resolution, avaible decoders etc.
	\end{itemize}
	
\subsubsection{Frameworks}
	\begin{itemize}
		\item Spring for Android\\
		Spring for Android is a framework bringing selected capablities from the Spring framework to the Android platform, allowing for faster, cleaner and more efficent developement of Andriod applications. Capabilities of Spring for Android include:
		\begin{itemize}
			\item A RESTful client with full Hypermedia as the Engine of Application State (HATEOAS) capability
			\item Authentication support which includes OAuth 1.0 and 2.0 capability, allow for token based authentication.
		\end{itemize}
	\end{itemize}

\subsubsection{Libraries}
	\begin{itemize}
		\item Android Butterknife\\
		Butterknife for Android brings Dependency Injection (DI) capability to Android's view system, allowing for cleaner and faster code. This is achieved by using Java annotations to bind visual components to corresponding Android Java components, thereby eliminating the need for poluting production source code with the infamous Android findViewById() method, which itself uses slow Java reflection to accomplish its task. Butterknife on the other hand produces code at compile time to assist in the lookup process.
	\end{itemize}

\subsubsection{Database System}
	\begin{itemize}
		\item Couchbase Mobile\\
		Couchbase Mobile is a NoSQL database system supporting all major mobile platforms, providing fast and consistent access to data, with or without a network connection. The Couchbase Mobile system consists of three key components:
		\begin{itemize}
			\item Couchbase Lite\\
			A mobile orientated NoSQL database allowing for low latency read/write access of data.
			\item Couchbase Sync Gateway\\
			An internet facing component, interfacing between mobile clients and the backend system to achieve secure synchronization of data.
			\item Couchbase Server\\
			A scalable, high performance, cluster based and enterprise based NoSQL database server providing NoSQL database functionality, with traditional based Structed Query Language functionality.
		\end{itemize}

		\item Utilizing a complete turn-around database system like Couchbase Mobile, will allow for the delivery of a fast and consistent user expierence, with the additional benefit of providing the mobile client with offline access and capability to view and edit selected data. 
	\end{itemize}

\subsubsection{Operating System}
\subsubsection{Dependency Management and Build Tools}



\end{document}
