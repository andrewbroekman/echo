\documentclass[a4paper,10pt]{article}

\usepackage[margin=2cm]{geometry}
\usepackage{graphicx}
\usepackage{hyperref}
\usepackage[all]{hypcap}
\usepackage{tabu}

\setlength{\parindent}{0pt}
\setlength{\parskip}{1ex plus 0.5ex minus 0.2ex}

\title{\includegraphics[width=12cm]{Eeufeeslogo.jpg} \\
       Software Requirements Specification \\ 
       and \\
       Technology Neutral Process Design \\
       Research Paper Management System \\
       \vspace{0.5cm}
       University of Pretoria \\
       \vspace{1.0cm}
       }

\date{} 
\author{Team Echo\\
	\vspace{0.5cm} \\
	\begin{tabu} to \textwidth { X[l] X[l]}
		\hline
		\textbf{Surname, First Name (Initial)}	& \textbf{Student Number}	\\ \hline \hline
		Bode, Elizabeth (EF)			& 14310156		\\ \hline
		Bondjobo, Jocelyn (JM)		& 13232852		\\ \hline
		Broekman, Andrew (A)		& 11089777		\\ \hline
		Loreggian, Fabio (FR)			& 14040426		\\ \hline
		Schutte, Gerome (GC)		& 12031519		\\ \hline
		Sefako, Motsitsiripe (MG)		& 12231097		\\ \hline
		Singh, Emilio (E)			& 14006512		\\ \hline
		\hline
	\end{tabu}}

\begin{document}
\clearpage
\maketitle
\thispagestyle{empty}

\newpage
\pagenumbering{roman}

\tableofcontents

\newpage
\pagenumbering{arabic}

\section{Vision}

\section{Background}

\section{Architecture requirements}
\subsection{Access channel requirements}

\subsection{Quality requirements}
\subsubsection{Performance}

\subsubsection{Reliability}
\begin{itemize}
\item To enable offline activity and access of data from the Android app, data synched must be made differentiable with timestamps.
\item The system must resolve synchronization conflicts using timestamps.
\item Hot swapping of system modules should not affect system service reliability.
\end{itemize}

\subsubsection{Scalability}

\subsubsection{Security}
\begin{itemize}
\item System should be resistant against SQL injections.
\item A user group hierarchy system should be used to manage user rights.
\item User credentials, i.e. username and password should not be stored on Android user device. A token based approach should be utilized so that only tokens are stored on end user device, allowing for the easy revocation of device access if a device is lost, such as an OAuth based security system could be utilized.
\end{itemize}

\subsubsection{Flexibility}
\begin{itemize}
\item The system should allow unregistered users to use existing third party web service login credentials to register or login to the service.
\item The system should be decoupled from the database technology it uses, and allow the client to select and change the database it uses in future.
\item Authentication mechanisms used should be decoupled from the system, allowing them to be interchangeablee.
\item Modules should be decoupled from one another, allowing the system to be extensible without a break in service, by integrating new modules and swapping out existing ones. 
\end{itemize}

\subsubsection{Maintainability}
\begin{itemize}
\item All code should be documented in the applicable language documentation framework, such as JavaDocs for a Java based system, DOxygen for a C/C++ based system, etc.
\item A coding style guide/manual should be setup and associated with the project, such that all developers use similar coding styles and conventions.
\item System should be separated in distinct, concise and independent modules to allow for easier maintenance.
\end{itemize}

\subsubsection{Auditability/Monitorability}
\begin{itemize}
\item Data in database should be always be consistent. This implies that all data should fulfill, constraints placed on the data by the data model, such as regex patterns, minimum and maximum length, not nullable fields, etc.
\item No data should ever be deleted.
\item It should always be possible to see which user created which objects at what time.
\item It should always be possible to see which user last changed objects and at what time.
\item An immutable log of user actions should be kept and should only be accessible by administrative staff.
\item Logging of stack traces and crash analytics should be implemented in the mobile client, to ensure the developers can see to client reliability.
\end{itemize}

\subsubsection{Integrability}
\begin{itemize}
\item The system should allow technology neutral importing and exporting of data.
\item The backend system should integrate with a desktop web client and Android mobile app clients.
\item The system should integrate with different backend authentication services.
\end{itemize}

\subsubsection{Cost}

\subsubsection{Usability}
\begin{itemize}
\item Each view in the desktop and mobile clients should be related to a single topic only.
\item The web client should render properly and be fully functional in modern web browsers.
\item Mobile devices running Android 4.2 and upwards should be fully supported.
\item Material design UI guidelines prescribed by Google must be used, to ensure that the clients feel modern and familiar. 
\end{itemize}

\subsection{Integration requirements}

\subsection{Architecture constraints}

\section{Functional requirements and application design}
\subsection{Use case prioritization}
\subsubsection{Critical}
\begin{itemize}
\item Item1
\item Item2
\end{itemize}

\subsubsection{Important}
\begin{itemize}
\item Item1
\item Item2
\end{itemize}

\subsubsection{Nice to have}
\begin{itemize}
\item Item1
\item Item2
\end{itemize}

\subsection{Use case/Services contracts}
\begin{tabu} to \textwidth { | X[l] | X[l] | X[l] | X[l] | }
	\hline
		Use Case		& Pre-Condition		& Post-Condition		& Description	\\ \hline \hline
		item 11		& item 12			& item13				& item 14  \\ \hline
		item 21		& item 22			& item23				& item 24  \\ \hline
		item 31		& item 32			& item33				& item 34  \\
	\hline
\end{tabu}

\subsection{Required functionality}

\subsection{Process specifications}

\subsection{Domain Model}

\section{Open Issues}
\subsection {GitHub Repository}
Team Echo Repository: \url{https://github.com/andrewbroekman/echo}

This repository contains:
\begin{enumerate}
\item All work done by team members.
\item CONTRIBUTORS.md file outlining which members where involved in which phases of the project.
\end{enumerate}
\end{document}
